O objetivo deste trabalho é apresentar as principais características do desenvolvimento móvel multiplataforma, comparando plataformas e \textit{frameworks} que irão auxiliar no processo de desenvolvimento de soluções compatíveis com os requisitos obrigatórios para que uma aplicação seja
portável em múltiplos sistemas operacionais móveis, ou seja, cosiderada uma aplicação multiplataforma. A metodologia aplicada à pesquisa é pautada diante da diversificação e consequentemente, da evolução crescente das tecnologias móveis, que denominaremos durante a pesquisa de "ecossistema móvel". 


O desenvolvimento de aplicativos móveis intensificou-se nos últimos anos devido ao crescimento acelerado e popularização dos smartphones. Cada empresa do ramo móvel possui seu próprio sistema operacional, loja de aplicativos, parcela de mercado e ambientes de desenvolvimento. No entanto, quanto mais sistemas diferentes existem, maior o esforço, custo e tempo para desenvolver um app para todas as plataformas existentes. Surgiu então, o conceito de desenvolvimento multiplataforma, com a premissa de codificar apenas uma vez e abranger várias plataformas. Nesse contexto, o presente trabalho busca definir vantagens e desvantagens da abordagem multiplataforma quando comparada com a abordagem nativa. Por meio da análise de um exemplo de uso, que consistiu na recriação de um aplicativo nativo, implementado originalmente para a plataforma iOS, utilizando o framework Ionic, foi possível comparar e comprovar empiricamente os dados obtidos na literatura. As ferramentas multiplataforma, atualmente, não possuem mais as limita ções apontadas pela literatura, pois evoluíram muito rapidamente ao longo dos últimos cinco anos. Concluiu-se, ao fim deste trabalho, que cada abordagem tem um momento certo para ser utilizada, não sendo uma melhor que a outra, mas apenas diferentes entre si, e deve-se avaliar cada caso, considerando-se uma série de fatores para escolher qual abordagem utilizar.

% Separe as palavras-chave por ponto
\palavraschave{ecossistema móvel. desenvolvimento. multiplataforma. 
ionic. nativescript}